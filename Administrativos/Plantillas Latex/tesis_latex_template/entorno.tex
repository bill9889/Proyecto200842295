%% ---------------------------------------------------------------------------
%% intro.tex
%%
%% Introduction
%%
%% $Id: intro.tex 1477 2010-07-28 21:34:43Z palvarado $
%% ---------------------------------------------------------------------------

\section*{Entorno del Proyecto}
\label{chp:entorno}

En el 2008 naci\'{o}, en Tecnol\'{o}gico de Costa Rica, el Laboratorio de Procesamiento de Se\~{n}ales e Im\'{a}genes  de la Escuela de Ingenier\'{i}a Electr\'{o}nica, con el objetivo solucionar problemas del \'{a}mbito nacional y regional, relacionados con procesamiento, an\'{a}lisis y reconocimiento de informaci\'{o}n transportada en se\~{n}ales temporales y espaciales.

En el SipLab se desarrollan proyectos de investigaci\'{o}n en los que participan profesores y estudiantes de las Escuelas de Ingenier\'{i}a Electr\'{o}nica, Computaci\'{o}n, Biolog\'{i}a, y Ingenier\'{i}a Forestal.

El objetivo general del laboratorio es solucionar problemas en diversas \'{a}reas como la medicina, ambiente y entrenamiento, por medio del desarrollo de algoritmos computacionales y plataformas de hardware especializadas que permitan la evaluaci\'{o}n y reconocimiento de im\'{a}genes, audio, se\~{n}ales el\'{e}ctricas y electromagn\'{e}ticas.

Un objetivo espec\'{i}fico del laboratorio es mejorar  la eficiencia sus desarrollos realizados mediante herramientas de hardware y software cada vez con mayor poder\'{i}o de procesamiento y bajo consumo de recursos energ\'{e}ticos.

Por ello el Laboratorio ha procurado adquirir herramientas especializadas como tarjetas de desarrollo, sistemas de aceleraci\'{o}n de hardware, software libre de alta confiabilidad, entre otras muchas herramientas, pero particularmente en Febrero de 2013 se adquiri\'{o} la tarjeta de desarrollo Finboard, con la cual se tiene pensado realizar pruebas de aplicaci\'{o}n de algoritmos de procesamiento de im\'{a}genes recientemente desarrollados por los investigadores.

Sin embargo en estos \'{u}ltimos meses no se ha podido realizar prueba alguna en la tarjeta debido a que los recursos de software de la tarjeta son limitados para implementaciones que tengan que ver con procesamientos digital de im\'{a}genes.

La FinBoard, es  una plataforma de hardware vers\'{a}til, con las herramientas necesarias de desarrollo de software para permitir la construcci\'{o}n de alto rendimiento, sistemas de visi\'{o}n integrados. Basado en el procesador Blackfin BF609 doble n\'{u}cleo de bajo costo, el kit es ideal para explorar el an\'{a}lisis de v\'{i}deo avanzado, simplificar y acelerar algoritmos de procesamiento de im\'{a}genes.

%%% Local Variables: 
%%% mode: latex
%%% TeX-master: "main"
%%% End: 
